%%%%%%%%%%%%%%%%%%%% author.tex %%%%%%%%%%%%%%%%%%%%%%%%%%%%%%%%%%%
%
% sample root file for your "contribution" to a contributed volume
%
% Use this file as a template for your own input.
%
%%%%%%%%%%%%%%%% Springer %%%%%%%%%%%%%%%%%%%%%%%%%%%%%%%%%%


% RECOMMENDED %%%%%%%%%%%%%%%%%%%%%%%%%%%%%%%%%%%%%%%%%%%%%%%%%%%
\documentclass[graybox]{svmult}

% choose options for [] as required from the list
% in the Reference Guide

\usepackage{mathptmx}       % selects Times Roman as basic font
\usepackage{helvet}         % selects Helvetica as sans-serif font
\usepackage{courier}        % selects Courier as typewriter font
\usepackage{type1cm}        % activate if the above 3 fonts are
                            % not available on your system
%
\usepackage{makeidx}         % allows index generation
\usepackage{graphicx}        % standard LaTeX graphics tool
                             % when including figure files
\usepackage{multicol}        % used for the two-column index
\usepackage[bottom]{footmisc}% places footnotes at page bottom

% see the list of further useful packages
% in the Reference Guide

\makeindex             % used for the subject index
                       % please use the style svind.ist with
                       % your makeindex program

%%%%%%%%%%%%%%%%%%%%%%%%%%%%%%%%%%%%%%%%%%%%%%%%%%%%%%%%%%%%%%%%%%%%%%%%%%%%%%%%%%%%%%%%%

\begin{document}

\title*{Metagenomic design and sequencing}
% Use \titlerunning{Short Title} for an abbreviated version of
% your contribution title if the original one is too long
\author{William L. Trimble, Stephanie M. Greenwald, Sarah Owens, anyone else who wants a piece }
% Use \authorrunning{Short Title} for an abbreviated version of
\authorrunning{Trimble, Greenwald, and Owens}
% your contribution title if the original one is too long
\institute{William L. Trimble, Stephanie M. Greenwald, and Sarah Owens  \at Institute for Genomics and Systems Biology, University of Chicago \email{name@email.address} } 
%   \and Name of Second Author \at Name, Address of Institute \email{name@email.address}}
%
% Use the package "url.sty" to avoid
% problems with special characters
% used in your e-mail or web address
%
\maketitle

% Please use the 'starred' version of the new Springer \texttt{abstract} command for typesetting the text of the online abstracts (cf. source file of this chapter template \texttt{abstract}) and include them with the source files of your manuscript. 
\abstract*{
Abstract summarizes the paper in a 10--15 lines for indexing, discovery, and marketing}

% Use the plain \texttt{abstract} command if the abstract is also to appear in the printed version of the book.}
\abstract{Abstract summarizes the paper in a 10--15 lines for indexing, discovery, and marketing (for print) } 

\section{The technology}
\label{sec:1}
Use the template \emph{chapter.tex} together with the Springer document class SVMono (monograph-type books) or SVMult (edited books) to style the various elements of your chapter content in the Springer layout.

Placeholder for platform recommendation table.

\section{Experimental design guidelines}
\label{sec:2}
% Always give a unique label
% and use \ref{<label>} for cross-references
% and \cite{<label>} for bibliographic references


\subsection{Sample replicates and controls}
\label{subsec:2}

\subsection{Design replication}
\label{subsec:2}

\subsection{Sample requirements }
\label{subsec:2}

\section{Bioinformatic analyses}
\label{sec:4}


\section{Effective results reporting }
\label{sec:5}

\section{Case study}
\label{sec:6}

% For figures use
%

\begin{figure}[t]
\sidecaption[t]
% Use the relevant command for your figure-insertion program
% to insert the figure file.
% For example, with the option graphics use
\includegraphics[scale=.65]{figure}
%
% If no graphics program available, insert a blank space i.e. use
%\picplace{5cm}{2cm} % Give the correct figure height and width in cm
%
%\caption{Please write your figure caption here}
\caption{If the width of the figure is less than 7.8 cm use the \texttt{sidecapion} command to flush the caption on the left side of the page. If the figure is positioned at the top of the page, align the sidecaption with the top of the figure -- to achieve this you simply need to use the optional argument \texttt{[t]} with the \texttt{sidecaption} command}
\label{fig:2}       % Give a unique label
\end{figure}

% Use the \index{} command to code your index words
%

\subsubsection{Subsubsection Heading}


%
\begin{acknowledgement}
If you want to include acknowledgments of assistance and the like at the end of an individual chapter please use the \verb|acknowledgement| environment -- it will automatically render Springer's preferred layout.
\end{acknowledgement}
%

%%%%%%%%%%%%%%%%%%%%%%%% referenc.tex %%%%%%%%%%%%%%%%%%%%%%%%%%%%%%
% sample references
% %
% Use this file as a template for your own input.
%
%%%%%%%%%%%%%%%%%%%%%%%% Springer-Verlag %%%%%%%%%%%%%%%%%%%%%%%%%%
%
% BibTeX users please use
% \bibliographystyle{}
% \bibliography{}
%

\begin{thebibliography}{99.}%
% and use \bibitem to create references.
%
%
% Use the following syntax and markup for your references if 
% the subject of your book is from the field 
% "Computer Science, Economics, Engineering, Geosciences, Life Sciences"
%
%
\bibitem{Brown} Brown B, Aaron M (2001) The politics of nature. In: Smith J (ed) The rise of modern genomics, 3rd edn. Wiley, New York 

\bibitem{MG-RAST}F Meyer, D Paarmann, M D'Souza, R Olson, EM Glass, M Kubal, T Paczian, A Rodriguez, R Stevens, A Wilke, J Wilkening, RA Edwards (2008). The metagenomics RAST server – a public resource for the automatic phylogenetic and functional analysis of metagenomes. BMC Bioinofmatics. 9:386. doi:10.1186/1471-2105-9-386

\bibitem{Mason-deepwater} Mason, O., Hazen T., Borglin, S., Chain, P., Dubinsky, E., Fortney, J., et al. (2012). Metagenome, metatranscriptome and single-cell sequencing reveal microbial response to Deepwater Horizon oil spill. The ISME journal, 6(9), 1715-1727

\bibitem{Korenblum} Korenblum E., Souza DB., Penna M, Seldin L. (2013). Molecular Analysis of the Bacterial Communities in Crude Oil Samples from Two Brazilian Offshore Petroleum Platforms. International Journal of Microbiology Volume 2012. doi:10.1155/2012/156537
Lorenz P and Eck J. (2005). Metagenomics Industrial Applications. Nature Reviews Microbiology 3, 510-516. doi:10.1038/nrmicro1161

\bibitem{Hampton-Marcell} Hampton-Marcell JT., Moormann SM., Owens SM., Gilbert JA. (2013) Preparation and Metatranscriptomic Analyses of Host-Microbe Systems in Microbial Metatgenomics, Metatranscrtomics, and Metaproteomics volume 531 issue 169. Academic Press. ISBN 978-0-12-407863-5. ISSN 0076-6879.

\bibitem{Rubin} Rubin BER., Gibbons SM., Kennedy S., Hampton-Marcell J., Owens S., Gilbert J. (2013) Investigating the Impact of Storage Conditions on Microbial Community Composition in Soil Samples. PLOSone. DOI: 10.1371/journal.pone.0070460

\bibitem{Segata} Segata, N., Boernigen D., Tickle TL., Morgan XC., Garrett WS, Huttenhower C. (2013) Computational Meta’omics for Microbial Community Studies. Molecular Systems Biology, doi: 10.1038/msb.2013.22

\bibitem{MIE} Thomas, T., Gilbert, J., Meyer, F. (2012). Metagenomics – a guide from sampling to data analysis. Microbial Informatics and Experimentation, 2, doi:10.1186/2042-5783-2-3

\bibitem{Wang}Wang LY., Ke WJ., Sun XB., Liu JF., Gu JD., Mu BZ. (2013). Comparison of bacterial community in aqueous and oil phases of water-flooded petroleum reservoirs using pyrosequencing and clone library approaches. Appl Microbiol Biotechnol. DOI 10.1007/s00253-013-5472-y

\bibitem{API} A Wilke, J Bischof, T Harrison, T Brettin, M D'Souza, W Gerlach, HMatthews, T Paczian, J Wilkening, E M Glass, N Desai, F Meyer. (2015) A RESTful API for Accessing Microbial Community Data for MG-RAST. PLOS: Computational Biology. DOI: 10.1371/journal.pcbi.1004008
%
% Online Document
%\bibitem{basic-online} Dod J (1999) Effective Substances. In: The dictionary of substances and their effects. Royal Society of Chemistry. Available via DIALOG. \\
%\url{http://www.rsc.org/dose/title of subordinate document. Cited 15 Jan 1999}
%
% Journal article by DOI
%\bibitem{basic-DOI} Slifka MK, Whitton JL (2000) Clinical implications of dysregulated cytokine production. J Mol Med, doi: 10.1007/s001090000086
%
% Journal article
%\bibitem{basic-journal} Smith J, Jones M Jr, Houghton L et al (1999) Future of health insurance. N Engl J Med 965:325--329
%
% Monograph
%\bibitem{basic-mono} South J, Blass B (2001) The future of modern genomics. Blackwell, London 
%
\end{thebibliography}

\end{document}
